


%%%% 1. DOCUMENTCLASS %%%%
\documentclass[journal=tosc,final]{iacrtrans}
%%%% NOTES:
% - Change "journal=tosc" to "journal=tches" if needed
% - Change "submission" to "final" for final version
% - Add "spthm" for LNCS-like theorems


%%%% 2. PACKAGES %%%%


%%%% 3. AUTHOR, INSTITUTE %%%%
\author{Moritz Rupp}
\institute{
  Hochschule Albstadt-Sigmaringen, Albstadt, Germany, \email{ruppmori@hs-albsig.de}
  
}
%%%% NOTES:
% - We need a city name for indexation purpose, even if it is redundant
%   (eg: University of Atlantis, Atlantis, Atlantis)
% - \inst{} can be omitted if there is a single institute,
%   or exactly one institute per author


%%%% 4. TITLE %%%%
\title{Fuzzing Methods}

\author{Moritz Rupp}
%%%% NOTES:
% - If the title is too long, or includes special macro, please
%   provide a "running title" as optional argument: \title[Short]{Long}
% - You can provide an optional subtitle with \subtitle.

\begin{document}

\maketitle
\author


%%% 5. KEYWORDS %%%%
\keywords{Offensive Security \and Fuzzing \and Brute-Forcing \and It-Security \and Testing }


%%%% 6. ABSTRACT %%%%
\begin{abstract}
Software testing has become a ever more important security method. Since modern Software becomes increasingly complex it's not possible to rely fully on manuel static testing. Fuzzing is a dynamic automated way of testing and sees growing usage among security proffesionals. This paper examines different apporaches to use this technologys. 
\end{abstract}
\tableofcontents
\newpage

%%%% 7. PAPER CONTENT %%%%
\section{Introduction}
Development of any kind has always come with bugs, errors and unintentionally behaviour. Software is no exception. Increasingly complex programs and growing technology stacks contribute to this problem. On top of that, finished components are often continuously integrated. As a result, it becomes ever more difficult to predict final program behaviour. This yields in risks, particular if an application is processing external data input. If not handled correctly it can pose security vulnerabilitys or data breaches. The bigger a programm gets, more often such unwanted appearances occur. A field in which this is most noticable is web-development. Whereas 10 years ago, most websites were built with the help of one or two technologys, modern web-applications often use several framework with huge amounts of dependencys. This lead to an increasing amount of security bugs. Different approaches in software testing tries to oppose that development. Back when software was relativly simple and free of dependencys, manuel testing was the state of play. That included code reviews and manuel checks for potentiell exploitation. This was time consuming and needed experts for every specific application. Therefore static analyis was quickly adapted. This contained new techniques such as pattern search with a control flow graph, data dependency graph and data flow analysis. When software grew even more in complexity, security researchers needed a more scalable approach. Hence dynamic analyis was implemented. This method tries to understand software behaviour by inputing different types of data. Fuzz-testing is the cutting edge of this technique and will be explored in this paper. At first we will examine the basic concept and functionality. Following different methods in Fuzzing will be explored. Finally we will have a conclusion that covers the pros and cons of Fuzz-testing. 
\section{What is Fuzzing?}
Fuzz testing or fuzzing is a method of testing software to detect security holes in applications, operating systems, and networks. It involves flooding their input interfaces with random data, called fuzz, to make them crash or trigger unexpected programm behaviour.\footnote{Source 1} Furthermore the computation results are monitered and reported, all in an automated way. Fuzzing has become increasingly more relevant among security researchers and is used by almost all big tech companys for penetration testing. Tech leaders find more than 80\% of their bugs by using different fuzzing Methods.\footnote{Source 2} For example Google has found over 19 thousend errors in their browser projekt chrome.\footnote{Source 3} Microsoft used dynamic analysis to test one of their flagship projects 'office' and found nearly 2000 bugs using the american fuzzing loop(afl). Also open source project such linux use Fuzzing methods on a great scale. Many security bugs within the Linux Kernel have been found this way.\\
Generally a distinction is made on how fuzzing is executed on a specific interface. Fuzzing threats software most of the time from a black box model. 
Mutation based , generation based fuzzing, Protocol based fuzzing. types of fuzzing: application fuzzing, protocol fuzzing, file format fuzzing, network fuzzing. 
\newpage
\section{Functionality}
3 things are needed, Fuzz-engine.
\section{Fuzzing Methods}
\dots
\subsection{Application Fuzzing}
\dots
\subsection{Protocoll Fuzzing}
\subsection{File format Fuzzing}
\subsection{Network Fuzzing}
\section{Conclusion}



%%%% 8. BILBIOGRAPHY %%%%
\bibliographystyle{alpha}
\bibliography{abbrev3,crypto,biblio}
%%%% NOTES
% - Download abbrev3.bib and crypto.bib from https://cryptobib.di.ens.fr/
% - Use bilbio.bib for additional references not in the cryptobib database.
%   If possible, take them from DBLP.

\end{document}
